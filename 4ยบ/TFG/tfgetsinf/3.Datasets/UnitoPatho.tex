\subsection{Unitopatho}

El conjunto de datos Unitopatho, presentado en \cite{Barbano_2021} está formado por 292 imágenes etiquetadas de parches imágenes teñidos con Hematoxilina y Eosina (H\&E). Está destinado a la formación de redes neuronales profundas para la clasificación de pólipos colorrectales y la clasificación de adenmoas.

Las imágenes fueron tomadas a través de un escáner Hamamatsu Nanozoomer S210 con una ampliación de 20x (0.4415 $\mu m/px$) y cada imagen, etiquetada por patólogos expertos, pertenece a un paciente diferente y a una de entre seis clases diferentes:

\begin{table}[!ht]
    \centering
    \begin{tabular}{|l|l|}
    \hline
        \textbf{Etiqueta} & \textbf{Descripción} \\ \hline
        \textbf{NORM} & Tejido normal \\ \hline
        \textbf{HP} & Pólipo hiperplásico \\ \hline
        \textbf{TA.HG} & Adenoma tubular, Displasia de alto grado \\ \hline
        \textbf{TA.LG} & Adenoma tubular, Displasia de bajo grado \\ \hline
        \textbf{TVA.HG} & Adenoma tubulovelloso, Displasia de alto grado \\ \hline
        \textbf{TVA.LG} & Adenoma tubulovelloso, Displasia de bajo grado \\ \hline
    \end{tabular}
    \caption{Etiquetas dataset Unitopatho}
    \label{tab:Unitopatho_labels}
\end{table}


